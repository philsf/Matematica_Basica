\everymath{\displaystyle}
%\documentclass[pdftex,a4paper]{article}
\documentclass[a4paper]{article}
%%classes: article, report, book, proc, amsproc

%%%%%%%%%%%%%%%%%%%%%%%%
%% Misc
% para acertar os acentos
\usepackage[brazilian]{babel} 
%\usepackage[portuguese]{babel} 
% \usepackage[english]{babel}
% \usepackage[T1]{fontenc}
% \usepackage[latin1]{inputenc}
\usepackage[utf8]{inputenc}
\usepackage{indentfirst}
\usepackage{fullpage}
% \usepackage{graphicx} %See PDF section
\usepackage{multicol}
\setlength{\columnseprule}{0.5pt}
\setlength{\columnsep}{20pt}
%%%%%%%%%%%%%%%%%%%%%%%%
%%%%%%%%%%%%%%%%%%%%%%%%
%% PDF support

\usepackage[pdftex]{color,graphicx}
% %% Hyper-refs
\usepackage[pdftex]{hyperref} % for printing
% \usepackage[pdftex,bookmarks,colorlinks]{hyperref} % for screen

%% \newif\ifPDF
%% \ifx\pdfoutput\undefined\PDFfalse
%% \else\ifnum\pdfoutput > 0\PDFtrue
%%      \else\PDFfalse
%%      \fi
%% \fi

%% \ifPDF
%%   \usepackage[T1]{fontenc}
%%   \usepackage{aeguill}
%%   \usepackage[pdftex]{graphicx,color}
%%   \usepackage[pdftex]{hyperref}
%% \else
%%   \usepackage[T1]{fontenc}
%%   \usepackage[dvips]{graphicx}
%%   \usepackage[dvips]{hyperref}
%% \fi

%%%%%%%%%%%%%%%%%%%%%%%%


%%%%%%%%%%%%%%%%%%%%%%%%
%% Math
\usepackage{amsmath,amsfonts,amssymb}
% para usar R de Real do jeito que o povo gosta
\usepackage{amsfonts} % \mathbb
% para usar as letras frescas como L de Espaco das Transf Lineares
% \usepackage{mathrsfs} % \mathscr

% Oferecer seno e tangente em pt, com os comandos usuais.
\providecommand{\sin}{} \renewcommand{\sin}{\hspace{2pt}\mathrm{sen}}
\providecommand{\tan}{} \renewcommand{\tan}{\hspace{2pt}\mathrm{tg}}

% dt of integrals = \ud t
\newcommand{\ud}{\mathrm{\ d}}
%%%%%%%%%%%%%%%%%%%%%%%%



\begin{document}

%%%%%%%%%%%%%%%%%%%%%%%%
%% Título e cabeçalho
%\noindent\parbox[c]{.15\textwidth}{\includegraphics[width=.15\textwidth]{logo}}\hfill
\parbox[c]{.825\textwidth}{\raggedright%
  \sffamily {\LARGE


Lista: Potenciação, Equações exponenciais, Funções
exponenciais e Logaritmos

\par\bigskip}
%{Estácio -- PRONATEC\par} 
%{Curso: Informática / Informática para Internet\par}
{Prof: Felipe Figueiredo\par}
{\url{http://sites.google.com/site/proffelipefigueiredo}}
}

Versão: \verb|20141124|

%%%%%%%%%%%%%%%%%%%%%%%%


%%%%%%%%%%%%%%%%%%%%%%%%
\section{Formulário}

\subsection{Potenciação}
Propriedades de potenciação

\begin{eqnarray*}
a^b a^c &= a^{b+c}\\
\frac{a^b}{a^c} &= a^{b-c}\\
(a^b)^c &= a^{bc}\\
\sqrt[c]{a^b} &= a^{\frac{b}{c}}\\
a^{-1} &= \frac{1}{a}\\
a^{-b} &= \frac{1}{a^b}
\end{eqnarray*}

Onde $a\in \mathbb{R}, b\in \mathbb{R}, c\in \mathbb{R}$, $a\ge0$.

\subsection{Equações exponenciais}

\begin{displaymath}
  a^x = a^y \Leftrightarrow x = y
\end{displaymath}

Onde $a\in \mathbb{R}, x\in \mathbb{R}, y\in \mathbb{R}$, $a>0$ e $a
\ne 1$

\subsection{Logaritmos}

Definição
\begin{displaymath}
  \log_b a = x \Leftrightarrow b^x = a
\end{displaymath}

Propriedades de logaritmos

\begin{eqnarray*}
  \log_b b &= 1\\
  \log_b 1 &= 0\\
  \log_b (x+y) &= \log_b x + \log_b y\\
  \log_b (a^x) &= x \log_b a
\end{eqnarray*}

Onde $b\in \mathbb{R}, x\in \mathbb{R}, y\in \mathbb{R}$, $b>0$ e $b
\ne 1$


\section{Exercícios}

\begin{enumerate}
\item Simplifique as seguintes expressões como potências

  \begin{enumerate}
  \item $3^4 \cdot 3^2$
  \item $5^2 \cdot 5$
  \item $2^5 \cdot 2^2$
  \item $\frac{2^5}{2^2}$
  \item $\frac{3^7}{3^2}$
  \item $\frac{7^2}{7^{10}}$
  \item $(2^3)^2$
  \item $(5^4)^5$
  \item $(1^{17})^2$
  \item $3^{-1}$
  \item $5^{-2}$
  \item $\frac{1}{2^{-3}}$
  \item $\frac{3}{3^{-1}}$
  \item $\frac{3^2 \cdot 3^3}{3^5}$
  \item $\frac{5^4}{5^{-5}}$
  \item $\frac{3^{5} \cdot 3^{-1}}{3^2 \cdot 3^{-3}}$
  \item $\frac{2^{-11} \cdot 2^{9}}{2^{12} \cdot 2^{-14} }$
  \end{enumerate}

\item Resolva as seguintes equações exponenciais e encontre o valor de
  $x$.
  \begin{enumerate}
  \item $2^x = 32$
  \item $2^x = 64$
  \item $3^x = \sqrt{81}$
  \item $5^x = \sqrt[3]{25}$
  \item $5^x = \sqrt{\frac{1}{5}}$
  \item $5^x = \sqrt[4]{\frac{1}{25}}$
  \item $7^x = \frac{1}{49}$
  \item $4^x = 2$
  \item $2 \cdot 2^{x} = \sqrt{2}$
  \item $3^{x} = \sqrt{\frac{1}{3}}$
  \item $\frac{5}{5^x} = 1$
  \item $(\frac{1}{5})^x = 25$
  \item $(\frac{1}{9})^x = 27$
  \end{enumerate}

\item Esboce o gráfico de cada uma das seguintes funções
  \begin{enumerate}
  \item $f(x) = 2^x$
  \item $f(x) = 3^x$
  \item $f(x) = 2^{2x}$
  \end{enumerate}

\item Calcule os seguintes logaritmos
  \begin{enumerate}
  \item $\log_7 7$
  \item $\log_3 1$
  \item $\log_6 1$
  \item $\log_3 9$
  \item $\log_2 \left(\frac{1}{32}\right)$
  \item $\log_3 \left(\frac{1}{81}\right)$
  \item $\log_2 \left(\frac{2}{64}\right)$
  \item $\log_7 \left(\sqrt[3]{49}\right)$
  \item $\log_3 \left(\sqrt[4]{\frac{1}{81}}\right)$
  \item $\log_\frac{1}{2} 2$
  \item $\log_\frac{1}{9} 27$
  \end{enumerate}

\end{enumerate}

\end{document}
