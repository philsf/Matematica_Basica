%\documentclass[pdftex,a4paper]{article}
\documentclass[a4paper]{article}
%%classes: article, report, book, proc, amsproc

%%%%%%%%%%%%%%%%%%%%%%%%
%% Misc
% para acertar os acentos
\usepackage[brazilian]{babel} 
%\usepackage[portuguese]{babel} 
% \usepackage[english]{babel}
% \usepackage[T1]{fontenc}
% \usepackage[latin1]{inputenc}
\usepackage[utf8]{inputenc}
\usepackage{indentfirst}
\usepackage{fullpage}
% \usepackage{graphicx} %See PDF section
\usepackage{multicol}
\setlength{\columnseprule}{0.5pt}
\setlength{\columnsep}{20pt}
%%%%%%%%%%%%%%%%%%%%%%%%
%%%%%%%%%%%%%%%%%%%%%%%%
%% PDF support

\usepackage[pdftex]{color,graphicx}
% %% Hyper-refs
\usepackage[pdftex]{hyperref} % for printing
% \usepackage[pdftex,bookmarks,colorlinks]{hyperref} % for screen

%% \newif\ifPDF
%% \ifx\pdfoutput\undefined\PDFfalse
%% \else\ifnum\pdfoutput > 0\PDFtrue
%%      \else\PDFfalse
%%      \fi
%% \fi

%% \ifPDF
%%   \usepackage[T1]{fontenc}
%%   \usepackage{aeguill}
%%   \usepackage[pdftex]{graphicx,color}
%%   \usepackage[pdftex]{hyperref}
%% \else
%%   \usepackage[T1]{fontenc}
%%   \usepackage[dvips]{graphicx}
%%   \usepackage[dvips]{hyperref}
%% \fi

%%%%%%%%%%%%%%%%%%%%%%%%


%%%%%%%%%%%%%%%%%%%%%%%%
%% Math
\usepackage{amsmath,amsfonts,amssymb}
% para usar R de Real do jeito que o povo gosta
\usepackage{amsfonts} % \mathbb
% para usar as letras frescas como L de Espaco das Transf Lineares
% \usepackage{mathrsfs} % \mathscr

% Oferecer seno e tangente em pt, com os comandos usuais.
\providecommand{\sin}{} \renewcommand{\sin}{\hspace{2pt}\mathrm{sen}}
\providecommand{\tan}{} \renewcommand{\tan}{\hspace{2pt}\mathrm{tg}}

% dt of integrals = \ud t
\newcommand{\ud}{\mathrm{\ d}}
%%%%%%%%%%%%%%%%%%%%%%%%



\begin{document}

%%%%%%%%%%%%%%%%%%%%%%%%
%% Título e cabeçalho
%\noindent\parbox[c]{.15\textwidth}{\includegraphics[width=.15\textwidth]{logo}}\hfill
\parbox[c]{.825\textwidth}{\raggedright%
  \sffamily {\LARGE

Lista: Funções do segundo grau

\par\bigskip}
%{Estácio -- PRONATEC\par} 
%{Curso: Informática / Informática para Internet\par}
{Prof: Felipe Figueiredo\par}
{\url{http://sites.google.com/site/proffelipefigueiredo}}

\vspace{1cm}
}
%%%%%%%%%%%%%%%%%%%%%%%%


%%%%%%%%%%%%%%%%%%%%%%%%

\section{Fomulário}

\begin{displaymath}
  y = a x^2 + b x + c
\end{displaymath}

\begin{itemize}
\item Discriminante ($\Delta$)
  \begin{displaymath}
    \Delta = b^2 - 4ac
  \end{displaymath}

\item Raízes reais
  \begin{displaymath}
    x = \frac{-b \pm \sqrt{\Delta}}{2a}
  \end{displaymath}

\item Vértice da parábola
  \begin{displaymath}
    V = (V_x, V_y)
  \end{displaymath}
  \begin{displaymath}
  V_x = \frac{-b}{2a}    
  \end{displaymath}
  \begin{displaymath}
    V_y = \frac{-\Delta}{4a}
  \end{displaymath}

\end{itemize}

\section{Exercícios}

Siga os passos encadeados 1--5 para esboçar o gráfico de cada uma das
funções quadráticas do item 6.

\begin{enumerate}
\item Determine a concavidade das funções do item 6
\item Determine o discriminante das funções do item 6
\item Encontre as raízes reais das funções do item 6, se houver
\item Determine o vértice das parábolas do item 6
\item Esboce o gráfico das funções do item 6
\item Seguem as funções quadráticas que serão analisadas nos itens
  1--5

  \begin{multicols}{2}
    
  \begin{enumerate}
  \item $y = x^2$
  \item $y = -x^2$
  \item $y = x^2 -1$
  \item $y = 1 - x^2$
  \item $y = x^2 + 1$
  \item $y = -x^2 + 2x$
  \item $y = 4x^2 + 4x$
  \item $y = -x^2 - 2x$
  \item $y = 4x^2 +x$
  \item $y = -x^2 +2x - 1$
  \item $y = x^2 -2x + 2$
  \item $y = \frac{1}{2}x^2 - \frac{1}{2}$
  \item $y = -x^2 +x - \frac{1}{4}$
  \item $y = \frac{1}{2}x^2 + \frac{1}{2}x + \frac{1}{8}$
  \item $y = -\frac{1}{4}x^2 -\frac{1}{2}x +2$
  \item $y = -x^2 + \sqrt{3}x$
  \item $y = x^2 + \sqrt{8}x + 4$
  \item $y = -x^2 + 2\sqrt{3}x - 10$
  \end{enumerate}

\end{multicols}

\end{enumerate}

\end{document}
