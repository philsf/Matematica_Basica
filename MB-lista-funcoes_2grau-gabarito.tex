%\documentclass[pdftex,a4paper]{article}
\documentclass[a4paper]{article}
%%classes: article, report, book, proc, amsproc

%%%%%%%%%%%%%%%%%%%%%%%%
%% Misc
% para acertar os acentos
\usepackage[brazilian]{babel} 
%\usepackage[portuguese]{babel} 
% \usepackage[english]{babel}
% \usepackage[T1]{fontenc}
% \usepackage[latin1]{inputenc}
\usepackage[utf8]{inputenc}
\usepackage{indentfirst}
\usepackage{fullpage}
% \usepackage{graphicx} %See PDF section
\usepackage{multicol}
\setlength{\columnseprule}{0.5pt}
\setlength{\columnsep}{20pt}
%%%%%%%%%%%%%%%%%%%%%%%%
%%%%%%%%%%%%%%%%%%%%%%%%
%% PDF support

\usepackage[pdftex]{color,graphicx}
% %% Hyper-refs
\usepackage[pdftex]{hyperref} % for printing
% \usepackage[pdftex,bookmarks,colorlinks]{hyperref} % for screen

%% \newif\ifPDF
%% \ifx\pdfoutput\undefined\PDFfalse
%% \else\ifnum\pdfoutput > 0\PDFtrue
%%      \else\PDFfalse
%%      \fi
%% \fi

%% \ifPDF
%%   \usepackage[T1]{fontenc}
%%   \usepackage{aeguill}
%%   \usepackage[pdftex]{graphicx,color}
%%   \usepackage[pdftex]{hyperref}
%% \else
%%   \usepackage[T1]{fontenc}
%%   \usepackage[dvips]{graphicx}
%%   \usepackage[dvips]{hyperref}
%% \fi

%%%%%%%%%%%%%%%%%%%%%%%%


%%%%%%%%%%%%%%%%%%%%%%%%
%% Math
\usepackage{amsmath,amsfonts,amssymb}
% para usar R de Real do jeito que o povo gosta
\usepackage{amsfonts} % \mathbb
% para usar as letras frescas como L de Espaco das Transf Lineares
% \usepackage{mathrsfs} % \mathscr

% Oferecer seno e tangente em pt, com os comandos usuais.
\providecommand{\sin}{} \renewcommand{\sin}{\hspace{2pt}\mathrm{sen}}
\providecommand{\tan}{} \renewcommand{\tan}{\hspace{2pt}\mathrm{tg}}

% dt of integrals = \ud t
\newcommand{\ud}{\mathrm{\ d}}
%%%%%%%%%%%%%%%%%%%%%%%%



\begin{document}

%%%%%%%%%%%%%%%%%%%%%%%%
%% Título e cabeçalho
%\noindent\parbox[c]{.15\textwidth}{\includegraphics[width=.15\textwidth]{logo}}\hfill
\parbox[c]{.825\textwidth}{\raggedright%
  \sffamily {\LARGE

Gabarito: Funções do segundo grau

\par\bigskip}
%{Estácio -- PRONATEC\par} 
%{Curso: Informática / Informática para Internet\par}
{Prof: Felipe Figueiredo\par}
{\url{http://sites.google.com/site/proffelipefigueiredo}}
}

Versão: \verb|20141124|

%%%%%%%%%%%%%%%%%%%%%%%%

\begin{tabular}{|c|c|c|c|c|}
  \hline
  Função & Item 1 & Item 2 & Item 3 & Item 4\\
  \hline
  \hline
  a & para cima & $\Delta = 0$ & $x=0$ & $V=(0,0)$\\
  b & para baixo & $\Delta = 0$ & $x = 0$ & $V=(0,0)$\\
  c & para cima & $\Delta = 4$ & $x_1 = -1, x_2 = 1$ & $V=(0,-1)$\\
  d & para baixo & $\Delta = 4$ & $x_1 = -1,x_2 =  1$ & $V=(0,1)$\\
  e & para cima & $\Delta = -4$ & não possui raízes reais & $V=(0,1)$\\
  f & para baixo & $\Delta = 4$ & $x_1 =0, x_2 = 2$ & $V=(1,1)$\\
  g & para cima & $\Delta = 16$ & $x_1 -1= , x_2 = 0$ & $V=(-\frac{1}{2},-1)$\\
  h & para baixo & $\Delta = 4$ & $x_1 = 0, x_2 = -2$ & $V=(-1,1)$\\
  i & para cima & $\Delta = 1$ & $x_1 = -\frac{1}{4}, x_2 = 0$ & $V=(-\frac{1}{8},-\frac{1}{16})$\\
  j & para baixo & $\Delta = 0$ & $x = 1$ & $V=(1,0)$\\
  k & para cima & $\Delta = -4$ & não possui raízes reais & $V=(1,1)$\\
  l & para cima & $\Delta = 1$ & $x_1 = -1, x_2 = 1$ & $V=(0,-\frac{1}{2})$\\
  m & para baixo & $\Delta = 0$ & $x = \frac{1}{2}$ & $V=(\frac{1}{2},0)$\\
  n & para cima & $\Delta = 0$ & $x = -\frac{1}{2}$ & $V=(-\frac{1}{2},0)$\\
  o & para baixo & $\Delta = \frac{9}{4}$ & $x_1 = -4, x_2 = 2$ & $V=(-1,\frac{9}{4})$\\
  p & para baixo & $\Delta = 3$ & $x_1 = 0, x_2 = \sqrt{3}$ & $V=(\frac{\sqrt{3}}{2},\frac{3}{4})$\\
  q & para cima & $\Delta = -8$ & não possui raízes reais & $V=(-\sqrt{2},2)$\\
  r & para baixo & $\Delta = -28$ & não possui raízes reais & $V=(\sqrt{3},-7)$\\
  \hline
    

\end{tabular}


%%%%%%%%%%%%%%%%%%%%%%%%


\end{document}
