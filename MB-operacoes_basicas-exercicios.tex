%\documentclass[pdftex,a4paper]{article}
\documentclass[a4paper]{article}
%%classes: article, report, book, proc, amsproc

%%%%%%%%%%%%%%%%%%%%%%%%
%% Misc
% para acertar os acentos
\usepackage[brazilian]{babel} 
%\usepackage[portuguese]{babel} 
% \usepackage[english]{babel}
% \usepackage[T1]{fontenc}
% \usepackage[latin1]{inputenc}
\usepackage[utf8]{inputenc}
\usepackage{indentfirst}
\usepackage{fullpage}
% \usepackage{graphicx} %See PDF section
\usepackage{multicol}
\setlength{\columnseprule}{0.5pt}
\setlength{\columnsep}{20pt}
%%%%%%%%%%%%%%%%%%%%%%%%
%%%%%%%%%%%%%%%%%%%%%%%%
%% PDF support

\usepackage[pdftex]{color,graphicx}
% %% Hyper-refs
\usepackage[pdftex]{hyperref} % for printing
% \usepackage[pdftex,bookmarks,colorlinks]{hyperref} % for screen

%% \newif\ifPDF
%% \ifx\pdfoutput\undefined\PDFfalse
%% \else\ifnum\pdfoutput > 0\PDFtrue
%%      \else\PDFfalse
%%      \fi
%% \fi

%% \ifPDF
%%   \usepackage[T1]{fontenc}
%%   \usepackage{aeguill}
%%   \usepackage[pdftex]{graphicx,color}
%%   \usepackage[pdftex]{hyperref}
%% \else
%%   \usepackage[T1]{fontenc}
%%   \usepackage[dvips]{graphicx}
%%   \usepackage[dvips]{hyperref}
%% \fi

%%%%%%%%%%%%%%%%%%%%%%%%


%%%%%%%%%%%%%%%%%%%%%%%%
%% Math
\usepackage{amsmath,amsfonts,amssymb}
% para usar R de Real do jeito que o povo gosta
\usepackage{amsfonts} % \mathbb
% para usar as letras frescas como L de Espaco das Transf Lineares
% \usepackage{mathrsfs} % \mathscr

% Oferecer seno e tangente em pt, com os comandos usuais.
\providecommand{\sin}{} \renewcommand{\sin}{\hspace{2pt}\mathrm{sen}}
\providecommand{\tan}{} \renewcommand{\tan}{\hspace{2pt}\mathrm{tg}}

% dt of integrals = \ud t
\newcommand{\ud}{\mathrm{\ d}}
%%%%%%%%%%%%%%%%%%%%%%%%



\begin{document}

%%%%%%%%%%%%%%%%%%%%%%%%
%% Título e cabeçalho
%\noindent\parbox[c]{.15\textwidth}{\includegraphics[width=.15\textwidth]{logo}}\hfill
\parbox[c]{.825\textwidth}{\raggedright%
  \sffamily {\LARGE

Lista: Conjuntos Numéricos e Operações Básicas

\par\bigskip}
{Prof: Felipe Figueiredo\par}
{\url{http://sites.google.com/site/proffelipefigueiredo}}
}

Versão: \verb|20141124|

%%%%%%%%%%%%%%%%%%%%%%%%


%%%%%%%%%%%%%%%%%%%%%%%%
\begin{enumerate}
\section{Exercícios}

\item Em cada questão abaixo os conjuntos A e B são iguais. Determine
  $m$ e $n$ em cada caso:
  \begin{enumerate}
  \item $A = \{1,2,3,m\}, B = \{n,2,3,4\}$
  \item $A = \{2,m,3\}, B = \{n,3,2\}$
  \item $A = \{1,1,2,2,3,m\}, B = \{3,2,1,n,3,1\}$
  \item $A = \{5,7,m\}, B = \{7,5,n\}$
  \end{enumerate}

\item Efetue as operações abaixo:
  \begin{enumerate}
  \item $3 \cdot (-5)$
  \item $(-3)\cdot(-2)\cdot(4)\cdot(-1)\cdot\frac{-10}{-8}$
  \item $5 \cdot \frac{2}{-5}$
  \item $\frac{-1}{2}\cdot\frac{2}{-3}$
  \item $\frac{1}{2}\cdot\frac{2}{3}\cdot\frac{3}{5}$
  \item $\frac{-2}{3}\cdot\frac{1}{-4}\cdot\frac{-3}{-2}$
  \item $\frac{4}{5} + 1$
  \item $\frac{3}{5} - 1$
  \item $\frac{1}{2} + \frac{5}{2}$
  \item $\frac{1}{2} + \frac{3}{2} - 1$
  \item $\frac{2}{5} + \frac{2}{3}$
  \item $\frac{1}{6} - \frac{2}{3}$
  \item $-3\cdot\left( 2 (-1) + 5 \cdot 2  \right)$
  \item $-2 \cdot \left( \frac{3}{2} - \frac{1}{4} \right)$
  \item $\frac{2}{-3}\left( \frac{-1}{-2} - 2 + \frac{-1}{2}  \right)$
  \item $\frac{\frac{1}{2}}{\frac{1}{3}}$
  \item $\frac{\frac{2}{5}}{\frac{3}{5}}$
  \item $\frac{\frac{1}{3} \cdot \frac{4}{-5}}{\frac{1}{2}}$
  \end{enumerate}

\item Efetue as operações com decimais
  \begin{enumerate}
  \item $3,2 + 1,8$
  \item $2,04 - 5,1$
  \item $1,1 \cdot 2$
  \item $2,1 \cdot 0,5$
  \item $\frac{1,5}{0,2}$ (sugestão: transforme em fração antes de
    efetuar a divisão)
  \item $\frac{1 - 0,75}{0,5}$
  \end{enumerate}

\section{Problemas}

\item O estádio Itaquerão, usado na abertura da Copa 2014 deveria ter
  originalmente 68.000 lugares, mas foi inaugurado com 62.600 pelos
  problemas nas obras. Quantos lugares deixaram de ser
  disponibilizados ao público?

\item Mariazinha trabalha como empregada doméstica e tem compulsão de
  consumo. Ela recebe o salário mínimo do estado do RJ no valor de R\$
  874,76, mas não conseguiu resistir a uma série de promoções de
  bolsas e sapatos gastou R\$ 924,76 no mesmo mês. Assumindo que ela
  começou o mês com saldo nulo, qual é o saldo da conta dela após
  receber o salário e fazer essas compras?

\item João comprou uma TV nova com muitas polegadas pagando 6
  prestações de R\$ 234,00. A mesma TV à vista custava R\$
  904,00. Quanto ele teria economizado se tivesse se programado para
  comprar essa TV à vista?

\item Pedro recebe R\$ 600 de salário base como professor primário do
  município de Pequenópolis. Além disso, ele recebe uma gratificação
  de $\frac{1}{6}$ do salário base para preparar aulas. Qual é o
  salário de Pedro?
\end{enumerate}

\end{document}
