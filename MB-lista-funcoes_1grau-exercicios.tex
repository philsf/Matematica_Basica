%\documentclass[pdftex,a4paper]{article}
\documentclass[a4paper]{article}
%%classes: article, report, book, proc, amsproc

%%%%%%%%%%%%%%%%%%%%%%%%
%% Misc
% para acertar os acentos
\usepackage[brazilian]{babel} 
%\usepackage[portuguese]{babel} 
% \usepackage[english]{babel}
% \usepackage[T1]{fontenc}
% \usepackage[latin1]{inputenc}
\usepackage[utf8]{inputenc}
\usepackage{indentfirst}
\usepackage{fullpage}
% \usepackage{graphicx} %See PDF section
\usepackage{multicol}
\setlength{\columnseprule}{0.5pt}
\setlength{\columnsep}{20pt}
%%%%%%%%%%%%%%%%%%%%%%%%
%%%%%%%%%%%%%%%%%%%%%%%%
%% PDF support

\usepackage[pdftex]{color,graphicx}
% %% Hyper-refs
\usepackage[pdftex]{hyperref} % for printing
% \usepackage[pdftex,bookmarks,colorlinks]{hyperref} % for screen

%% \newif\ifPDF
%% \ifx\pdfoutput\undefined\PDFfalse
%% \else\ifnum\pdfoutput > 0\PDFtrue
%%      \else\PDFfalse
%%      \fi
%% \fi

%% \ifPDF
%%   \usepackage[T1]{fontenc}
%%   \usepackage{aeguill}
%%   \usepackage[pdftex]{graphicx,color}
%%   \usepackage[pdftex]{hyperref}
%% \else
%%   \usepackage[T1]{fontenc}
%%   \usepackage[dvips]{graphicx}
%%   \usepackage[dvips]{hyperref}
%% \fi

%%%%%%%%%%%%%%%%%%%%%%%%


%%%%%%%%%%%%%%%%%%%%%%%%
%% Math
\usepackage{amsmath,amsfonts,amssymb}
% para usar R de Real do jeito que o povo gosta
\usepackage{amsfonts} % \mathbb
% para usar as letras frescas como L de Espaco das Transf Lineares
% \usepackage{mathrsfs} % \mathscr

% Oferecer seno e tangente em pt, com os comandos usuais.
\providecommand{\sin}{} \renewcommand{\sin}{\hspace{2pt}\mathrm{sen}}
\providecommand{\tan}{} \renewcommand{\tan}{\hspace{2pt}\mathrm{tg}}

% dt of integrals = \ud t
\newcommand{\ud}{\mathrm{\ d}}
%%%%%%%%%%%%%%%%%%%%%%%%



\begin{document}

%%%%%%%%%%%%%%%%%%%%%%%%
%% Título e cabeçalho
%\noindent\parbox[c]{.15\textwidth}{\includegraphics[width=.15\textwidth]{logo}}\hfill
\parbox[c]{.825\textwidth}{\raggedright%
  \sffamily {\LARGE

Lista: Funções do primeiro grau

\par\bigskip}
%{Estácio -- PRONATEC\par} 
{Prof: Felipe Figueiredo\par}
{\url{http://sites.google.com/site/proffelipefigueiredo}}

\vspace{1cm}
}
%%%%%%%%%%%%%%%%%%%%%%%%


%%%%%%%%%%%%%%%%%%%%%%%%

\section{Formulário}

\begin{itemize}
\item Função constante:

\begin{displaymath}
  f(x) = c
\end{displaymath}

onde $c \in \mathbb{R}$.

\item Função linear:

\begin{displaymath}
  f(x) = ax
\end{displaymath}

onde $a \in \mathbb{R}$.

\item Função afim:

\begin{displaymath}
  f(x) = ax +b
\end{displaymath}

onde $a \in \mathbb{R}, b \in \mathbb{R}$.

\item Raiz da função afim:

\begin{displaymath}
 f(x_0) = 0 \Rightarrow  x_0 = -\frac{b}{a}
\end{displaymath}

\end{itemize}


\section{Exercícios}

\begin{enumerate}

\item Esboce os gráficos das seguintes funções constantes:

  \begin{enumerate}
  \item $f(x) = 1$
  \item $f(x) = 3$
  \item $f(x) = -2$
  \item $f(x) = -5$
  \item $f(x) = 4$
  \end{enumerate}

\item Esboce no mesmo plano cartesiano os gráficos das seguintes
  funções lineares:

  \begin{enumerate}
  \item $f(x) = x$
  \item $f(x) = 2x$
  \item $f(x) = 3x$
  \item $f(x) = -x$
  \item $f(x) = -2x$
  \end{enumerate}

\item Para cada uma das funções abaixo, diga qual é o coeficiente
  angular e o coeficiente linear da função.

  \begin{enumerate}
  \item $f(x) = x+1$
  \item $f(x) = 2x+1$
  \item $f(x) = -x+1$
  \item $f(x) = -x-1$
  \item $f(x) = 5x-5$
  \item $f(x) = -2x +4$
  \item $f(x) = 2x -4$
  \end{enumerate}

\item Para cada uma das funções no exercício 3, encontre sua raiz
  real.

\item Esboce os gráficos das funções do exercício 3.

\item Para cada uma das funções no exercício 3, encontre o valor da
  função para $x=3$

% \section{Problemas}

% \item 
\end{enumerate}
\end{document}
