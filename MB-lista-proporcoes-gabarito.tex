%\documentclass[pdftex,a4paper]{article}
\documentclass[a4paper]{article}
%%classes: article, report, book, proc, amsproc

%%%%%%%%%%%%%%%%%%%%%%%%
%% Misc
% para acertar os acentos
\usepackage[brazilian]{babel} 
%\usepackage[portuguese]{babel} 
% \usepackage[english]{babel}
% \usepackage[T1]{fontenc}
% \usepackage[latin1]{inputenc}
\usepackage[utf8]{inputenc}
\usepackage{indentfirst}
\usepackage{fullpage}
% \usepackage{graphicx} %See PDF section
\usepackage{multicol}
\setlength{\columnseprule}{0.5pt}
\setlength{\columnsep}{20pt}
%%%%%%%%%%%%%%%%%%%%%%%%
%%%%%%%%%%%%%%%%%%%%%%%%
%% PDF support

\usepackage[pdftex]{color,graphicx}
% %% Hyper-refs
\usepackage[pdftex]{hyperref} % for printing
% \usepackage[pdftex,bookmarks,colorlinks]{hyperref} % for screen

%% \newif\ifPDF
%% \ifx\pdfoutput\undefined\PDFfalse
%% \else\ifnum\pdfoutput > 0\PDFtrue
%%      \else\PDFfalse
%%      \fi
%% \fi

%% \ifPDF
%%   \usepackage[T1]{fontenc}
%%   \usepackage{aeguill}
%%   \usepackage[pdftex]{graphicx,color}
%%   \usepackage[pdftex]{hyperref}
%% \else
%%   \usepackage[T1]{fontenc}
%%   \usepackage[dvips]{graphicx}
%%   \usepackage[dvips]{hyperref}
%% \fi

%%%%%%%%%%%%%%%%%%%%%%%%


%%%%%%%%%%%%%%%%%%%%%%%%
%% Math
\usepackage{amsmath,amsfonts,amssymb}
% para usar R de Real do jeito que o povo gosta
\usepackage{amsfonts} % \mathbb
% para usar as letras frescas como L de Espaco das Transf Lineares
% \usepackage{mathrsfs} % \mathscr

% Oferecer seno e tangente em pt, com os comandos usuais.
\providecommand{\sin}{} \renewcommand{\sin}{\hspace{2pt}\mathrm{sen}}
\providecommand{\tan}{} \renewcommand{\tan}{\hspace{2pt}\mathrm{tg}}

% dt of integrals = \ud t
\newcommand{\ud}{\mathrm{\ d}}
%%%%%%%%%%%%%%%%%%%%%%%%



\begin{document}

%%%%%%%%%%%%%%%%%%%%%%%%
%% Título e cabeçalho
%\noindent\parbox[c]{.15\textwidth}{\includegraphics[width=.15\textwidth]{logo}}\hfill
\parbox[c]{.825\textwidth}{\raggedright%
  \sffamily {\LARGE

Gabarito: Proporções

\par\bigskip}
%{Estácio -- PRONATEC\par} 
%{Curso: Informática / Informática para Internet\par}
{Prof: Felipe Figueiredo\par}
{\url{http://sites.google.com/site/proffelipefigueiredo}}

\vspace{1cm}
}
%%%%%%%%%%%%%%%%%%%%%%%%


%%%%%%%%%%%%%%%%%%%%%%%%

\begin{enumerate}
\section*{Problemas de regra de três}

\item % Identifique nos exercícios a seguir se as grandezas estão em
  % proporção direta ou inversa, e encontre a resposta de cada.

  \begin{enumerate}
  \item % Você caminha 4km em uma hora. Se você tiver que caminhar 12km,
    % quantas horas você levará?

  \item%  Um pintor pinta uma casa inteira em 21 horas. Em quanto tempo
    % 3 pintores pintarão a mesma casa?

  \item%  Um sapateiro conserta um tênis em 30min. Quanto tempo ele
    % levará para consertar 8 tênis?

  \item%  Uma mulher leva 40 minutos para se decidir entre 5 vestidos
    % diferentes. Quantos minutos ela precisaria se tivesse 8 vestidos?

  \item%  Um cafeicultor consegue colher 140kg de café em 8
    % horas. Quantas horas ele precisa para colher 320kg de café?

  \item%  Dois cafeicultores colhem 280kg de café em 8 horas. Quantas
    % horas 3 cafeicultores precisam para colher a mesma quantidade de
    % café?

  \item%  Uma mangueira enche uma piscina de borracha de 10.000L em 4
    % horas. Em quanto tempo essa mangueira deposita 6.000L nesta
    % piscina?
  \end{enumerate}

\section*{Exercícios de porcentagem}

\item %Encontre as seguintes porcentagens

  \begin{enumerate}
  \item %20\% de 100
  \item %30\% de 10
  \item %4\% de 25
  \item %25\% de 1
  \item %34\% de 200
  \item %40\% de 60\%
  \item %20\% de 70\%
  \item %75\% de 10\%
  \item %10\% de 5\% de 10
  \item %15\% de 50\% de 120
  \end{enumerate}

\end{enumerate}

\end{document}
