%\documentclass[pdftex,a4paper]{article}
\documentclass[a4paper]{article}
%%classes: article, report, book, proc, amsproc

%%%%%%%%%%%%%%%%%%%%%%%%
%% Misc
% para acertar os acentos
\usepackage[brazilian]{babel} 
%\usepackage[portuguese]{babel} 
% \usepackage[english]{babel}
% \usepackage[T1]{fontenc}
% \usepackage[latin1]{inputenc}
\usepackage[utf8]{inputenc}
\usepackage{indentfirst}
\usepackage{fullpage}
% \usepackage{graphicx} %See PDF section
\usepackage{multicol}
\setlength{\columnseprule}{0.5pt}
\setlength{\columnsep}{20pt}
%%%%%%%%%%%%%%%%%%%%%%%%
%%%%%%%%%%%%%%%%%%%%%%%%
%% PDF support

\usepackage[pdftex]{color,graphicx}
% %% Hyper-refs
\usepackage[pdftex]{hyperref} % for printing
% \usepackage[pdftex,bookmarks,colorlinks]{hyperref} % for screen

%% \newif\ifPDF
%% \ifx\pdfoutput\undefined\PDFfalse
%% \else\ifnum\pdfoutput > 0\PDFtrue
%%      \else\PDFfalse
%%      \fi
%% \fi

%% \ifPDF
%%   \usepackage[T1]{fontenc}
%%   \usepackage{aeguill}
%%   \usepackage[pdftex]{graphicx,color}
%%   \usepackage[pdftex]{hyperref}
%% \else
%%   \usepackage[T1]{fontenc}
%%   \usepackage[dvips]{graphicx}
%%   \usepackage[dvips]{hyperref}
%% \fi

%%%%%%%%%%%%%%%%%%%%%%%%


%%%%%%%%%%%%%%%%%%%%%%%%
%% Math
\usepackage{amsmath,amsfonts,amssymb}
% para usar R de Real do jeito que o povo gosta
\usepackage{amsfonts} % \mathbb
% para usar as letras frescas como L de Espaco das Transf Lineares
% \usepackage{mathrsfs} % \mathscr

% Oferecer seno e tangente em pt, com os comandos usuais.
\providecommand{\sin}{} \renewcommand{\sin}{\hspace{2pt}\mathrm{sen}}
\providecommand{\tan}{} \renewcommand{\tan}{\hspace{2pt}\mathrm{tg}}

% dt of integrals = \ud t
\newcommand{\ud}{\mathrm{\ d}}
%%%%%%%%%%%%%%%%%%%%%%%%



\begin{document}

%%%%%%%%%%%%%%%%%%%%%%%%
%% Título e cabeçalho
%\noindent\parbox[c]{.15\textwidth}{\includegraphics[width=.15\textwidth]{logo}}\hfill
\parbox[c]{.825\textwidth}{\raggedright%
  \sffamily {\LARGE

Gabarito: Funções do primeiro grau

\par\bigskip}
%{Estácio -- PRONATEC\par} 
%{Curso: Informática / Informática para Internet\par}
{Prof: Felipe Figueiredo\par}
{\url{http://sites.google.com/site/proffelipefigueiredo}}
}

Versão: \verb|20141124|

%%%%%%%%%%%%%%%%%%%%%%%%


%%%%%%%%%%%%%%%%%%%%%%%%
\begin{enumerate}

\section{}

\section{}

\item %gráfico função constante

\item %gráfico função linear

\item %coeficientes
  \begin{enumerate}
  \item $a=1, b=1$
  \item $a=2, b=1$
  \item $a=-1, b=1$
  \item $a=-1, b=-1$
  \item $a=5, b=5$
  \item $a=-2, b=4$
  \item $a=2, b=-4$
  \end{enumerate}
\item %raízes função afim

  \begin{enumerate}
  \item $x= -1$ %$f(x) = x+1$
  \item $x= -\frac{1}{2}$ %$f(x) = 2x+1$
  \item $x= 1$ %$f(x) = -x+1$
  \item $x= -1$ %$f(x) = -x-1$
  \item $x= 1$ %$f(x) = 5x-5$
  \item $x= 2$ %$f(x) = -2x +4$
  \item $x= 2$ %$f(x) = 2x -4$
  \end{enumerate}

\item %gráfico função afim

\item % f(3)
  \begin{enumerate}
  \item $f(3) = 4$
  \item $f(3) = 7$
  \item $f(3) = -2$
  \item $f(3) = -4$
  \item $f(3) = 10$
  \item $f(3) = -2$
  \item $f(3) = 2$
  \end{enumerate}

% \item 



\end{enumerate}

\end{document}
