%\documentclass[pdftex,a4paper]{article}
\documentclass[a4paper]{article}
%%classes: article, report, book, proc, amsproc

%%%%%%%%%%%%%%%%%%%%%%%%
%% Misc
% para acertar os acentos
\usepackage[brazilian]{babel} 
%\usepackage[portuguese]{babel} 
% \usepackage[english]{babel}
% \usepackage[T1]{fontenc}
% \usepackage[latin1]{inputenc}
\usepackage[utf8]{inputenc}
\usepackage{indentfirst}
\usepackage{fullpage}
% \usepackage{graphicx} %See PDF section
\usepackage{multicol}
\setlength{\columnseprule}{0.5pt}
\setlength{\columnsep}{20pt}
%%%%%%%%%%%%%%%%%%%%%%%%
%%%%%%%%%%%%%%%%%%%%%%%%
%% PDF support

\usepackage[pdftex]{color,graphicx}
% %% Hyper-refs
\usepackage[pdftex]{hyperref} % for printing
% \usepackage[pdftex,bookmarks,colorlinks]{hyperref} % for screen

%% \newif\ifPDF
%% \ifx\pdfoutput\undefined\PDFfalse
%% \else\ifnum\pdfoutput > 0\PDFtrue
%%      \else\PDFfalse
%%      \fi
%% \fi

%% \ifPDF
%%   \usepackage[T1]{fontenc}
%%   \usepackage{aeguill}
%%   \usepackage[pdftex]{graphicx,color}
%%   \usepackage[pdftex]{hyperref}
%% \else
%%   \usepackage[T1]{fontenc}
%%   \usepackage[dvips]{graphicx}
%%   \usepackage[dvips]{hyperref}
%% \fi

%%%%%%%%%%%%%%%%%%%%%%%%


%%%%%%%%%%%%%%%%%%%%%%%%
%% Math
\usepackage{amsmath,amsfonts,amssymb}
% para usar R de Real do jeito que o povo gosta
\usepackage{amsfonts} % \mathbb
% para usar as letras frescas como L de Espaco das Transf Lineares
% \usepackage{mathrsfs} % \mathscr

% Oferecer seno e tangente em pt, com os comandos usuais.
\providecommand{\sin}{} \renewcommand{\sin}{\hspace{2pt}\mathrm{sen}}
\providecommand{\tan}{} \renewcommand{\tan}{\hspace{2pt}\mathrm{tg}}

% dt of integrals = \ud t
\newcommand{\ud}{\mathrm{\ d}}
%%%%%%%%%%%%%%%%%%%%%%%%



\begin{document}

%%%%%%%%%%%%%%%%%%%%%%%%
%% Título e cabeçalho
%\noindent\parbox[c]{.15\textwidth}{\includegraphics[width=.15\textwidth]{logo}}\hfill
\parbox[c]{.825\textwidth}{\raggedright%
  \sffamily {\LARGE

Gabarito: Conjuntos Numéricos e Operações Básicas

\par\bigskip}
%{Estácio -- PRONATEC\par} 
%{Curso: Informática / Informática para Internet\par}
{Prof: Felipe Figueiredo\par}
{\url{http://sites.google.com/site/proffelipefigueiredo}}

\vspace{1cm}
}
%%%%%%%%%%%%%%%%%%%%%%%%


%%%%%%%%%%%%%%%%%%%%%%%%
\begin{enumerate}

\section{}


\item 
  \begin{enumerate}
  \item $m=4, n=1$
  \item $m=2, n=3$
  \item $m=3, n=2$
  \item $m=n$
  \end{enumerate}


\item 
  \begin{enumerate}
  \item $-15$
  \item $-30$
  \item $-2$
  \item $\frac{1}{3}$
  \item $\frac{1}{5}$
  \item $\frac{1}{4}$
  \item $\frac{9}{5}$
  \item $\frac{-2}{5}$
  \item $\frac{6}{2}=3$
  \item $\frac{2}{2}=1$
  \item $\frac{16}{15}$
  \item $\frac{-3}{6}=\frac{-1}{2}$
  \item $-24$
  \item $\frac{-5}{2}$
  \item $\frac{4}{3}$
  \item $\frac{3}{2}$
  \item $\frac{10}{15}=\frac{2}{3}$
  \item $\frac{-8}{15}$ 
  \end{enumerate}

\item 
  \begin{enumerate}
  \item $5,0$
  \item $-3,06$
  \item $2,2$
  \item $1,05$
  \item $7,5$
  \item $0,5$
  \end{enumerate}

\section{}

\item $68000 - 62600 = 5400$ lugares.

\item $874,76 - 924,76 = -50,00$ reais.

\item $900,00 - 6\cdot 234,00 = 500,00$ reais.

\item $600 + \frac{1}{6}\cdot 600 = 600\cdot\frac{7}{6} = 700$ reais.
\end{enumerate}

\end{document}
