\everymath{\displaystyle}
%\documentclass[pdftex,a4paper]{article}
\documentclass[a4paper]{article}
%%classes: article, report, book, proc, amsproc

%%%%%%%%%%%%%%%%%%%%%%%%
%% Misc
% para acertar os acentos
\usepackage[brazilian]{babel} 
%\usepackage[portuguese]{babel} 
% \usepackage[english]{babel}
% \usepackage[T1]{fontenc}
% \usepackage[latin1]{inputenc}
\usepackage[utf8]{inputenc}
\usepackage{indentfirst}
\usepackage{fullpage}
% \usepackage{graphicx} %See PDF section
\usepackage{multicol}
\setlength{\columnseprule}{0.5pt}
\setlength{\columnsep}{20pt}
%%%%%%%%%%%%%%%%%%%%%%%%
%%%%%%%%%%%%%%%%%%%%%%%%
%% PDF support

\usepackage[pdftex]{color,graphicx}
% %% Hyper-refs
\usepackage[pdftex]{hyperref} % for printing
% \usepackage[pdftex,bookmarks,colorlinks]{hyperref} % for screen

%% \newif\ifPDF
%% \ifx\pdfoutput\undefined\PDFfalse
%% \else\ifnum\pdfoutput > 0\PDFtrue
%%      \else\PDFfalse
%%      \fi
%% \fi

%% \ifPDF
%%   \usepackage[T1]{fontenc}
%%   \usepackage{aeguill}
%%   \usepackage[pdftex]{graphicx,color}
%%   \usepackage[pdftex]{hyperref}
%% \else
%%   \usepackage[T1]{fontenc}
%%   \usepackage[dvips]{graphicx}
%%   \usepackage[dvips]{hyperref}
%% \fi

%%%%%%%%%%%%%%%%%%%%%%%%


%%%%%%%%%%%%%%%%%%%%%%%%
%% Math
\usepackage{amsmath,amsfonts,amssymb}
% para usar R de Real do jeito que o povo gosta
\usepackage{amsfonts} % \mathbb
% para usar as letras frescas como L de Espaco das Transf Lineares
% \usepackage{mathrsfs} % \mathscr

% Oferecer seno e tangente em pt, com os comandos usuais.
\providecommand{\sin}{} \renewcommand{\sin}{\hspace{2pt}\mathrm{sen}}
\providecommand{\tan}{} \renewcommand{\tan}{\hspace{2pt}\mathrm{tg}}

% dt of integrals = \ud t
\newcommand{\ud}{\mathrm{\ d}}
%%%%%%%%%%%%%%%%%%%%%%%%



\begin{document}

%%%%%%%%%%%%%%%%%%%%%%%%
%% Título e cabeçalho
%\noindent\parbox[c]{.15\textwidth}{\includegraphics[width=.15\textwidth]{logo}}\hfill
\parbox[c]{.825\textwidth}{\raggedright%
  \sffamily {\LARGE

Gabarito: Potenciação, Equações exponenciais, Funções
exponenciais e Logaritmos

\par\bigskip}
%{Estácio -- PRONATEC\par} 
%{Curso: Informática / Informática para Internet\par}
{Prof: Felipe Figueiredo\par}
{\url{http://sites.google.com/site/proffelipefigueiredo}}
}

Versão: \verb|20141213|

%%%%%%%%%%%%%%%%%%%%%%%%


%%%%%%%%%%%%%%%%%%%%%%%%

\begin{enumerate}
\item %Simplifique as seguintes expressões como potências

  \begin{enumerate}
  \item $3^6$ %$3^4 \cdot 3^2$
  \item $5^3$ %$5^2 \cdot 5$
  \item $2^7$ %$2^5 \cdot 2^2$
  \item $2^3$ %$\frac{2^5}{2^2}$
  \item $3^5$ %$\frac{3^7}{3^2}$
  \item $7^{-8} = \frac{1}{7^8}$ %$\frac{7^2}{7^{10}}$
  \item $2^6$ %$(2^3)^2$
  \item $5^{20}$ %$(5^4)^5$
  \item $1$ %$(1^{17})^2$
  \item $\frac{1}{3}$ %$3^{-1}$
  \item $\frac{1}{5^2}$ %$5^{-2}$
  \item $2^3$ %$\frac{1}{2^{-3}}$
  \item $3^2$ %$\frac{3}{3^{-1}}$
  \item $5^0=1$ %$\frac{3^2 \cdot 3^3}{3^5}$
  \item $5^9$ %$\frac{5^4}{5^{-5}}$
  \item $3^5$ %$\frac{3^{5} \cdot 3^{-1}}{3^2 \cdot 3^{-3}}$
  \item $2^0=1$ %$\frac{2^{-11} \cdot 2^{9}}{2^{12} \cdot 2^{-14} }$
  \end{enumerate}

\item % Resolva as seguintes equações exponenciais e encontre o valor de
  % $x$.
  \begin{enumerate}
  \item $x=5$ %$2^x = 32$
  \item $x=6$ %$2^x = 64$
  \item $x=2$ %$3^x = \sqrt{81}$
  \item $x=\frac{2}{3}$ %$5^x = \sqrt[3]{25}$
  \item $x=-\frac{1}{2}$ %$5^x = \sqrt{\frac{1}{5}}$
  \item $x=-\frac{1}{2}$ %$5^x = \sqrt[4]{\frac{1}{25}}$
  \item $x=-2$ %$7^x = \frac{1}{49}$
  \item $2x=1 \Rightarrow x=\frac{1}{2}$ %$4^x = 2$
  \item $x+1=\frac{1}{2} \Rightarrow x=\frac{3}{2}$ %$2 \cdot 2^{x} = \sqrt{2}$
  \item $x=-\frac{1}{2}$ %$3^{x} = \sqrt{\frac{1}{3}}$
  \item $1-x=0 \Rightarrow x=1$ %$\frac{5}{5^x} = 1$
  \item $-x = 2 \Rightarrow x=-2$ %$(\frac{1}{5})^x = 25$
  \item $-2x = 3 \Rightarrow x=-\frac{3}{2}$ %$(\frac{1}{9})^x = 27$
  \end{enumerate}

\newpage
\item % Esboce o gráfico de cada uma das seguintes funções
  \begin{enumerate}
  \item  %$f(x) = 2^x$
  \item  %$f(x) = 3^x$
  \item  %$f(x) = 2^{2x}$
  \end{enumerate}

\item %Calcule os seguintes logaritmos
  \begin{enumerate}
  \item $1$ %$\log_7 7$
  \item $0$ %$\log_3 1$
  \item $0$ %$\log_6 1$
  \item $2$ %$\log_3 9$
  \item $-5$ %$\log_2 \left(\frac{1}{32}\right)$
  \item $-4$ %$\log_3 \left(\frac{1}{81}\right)$
  \item $-5$ %$\log_2 \left(\frac{2}{64}\right)$
  \item $\frac{2}{3}$ %$\log_7 \left(\sqrt[3]{49}\right)$
  \item $-1$ %$\log_3 \left(\sqrt[4]{\frac{1}{81}}\right)$
  \item $-1$ %$\log_\frac{1}{2} 2$
  \item $-\frac{3}{2}$ %$\log_\frac{1}{9} 27$
  \end{enumerate}

\end{enumerate}

\end{document}
