%\documentclass[pdftex,a4paper]{article}
\documentclass[a4paper]{article}
%%classes: article, report, book, proc, amsproc

%%%%%%%%%%%%%%%%%%%%%%%%
%% Misc
% para acertar os acentos
\usepackage[brazilian]{babel} 
%\usepackage[portuguese]{babel} 
% \usepackage[english]{babel}
% \usepackage[T1]{fontenc}
% \usepackage[latin1]{inputenc}
\usepackage[utf8]{inputenc}
\usepackage{indentfirst}
\usepackage{fullpage}
% \usepackage{graphicx} %See PDF section
\usepackage{multicol}
\setlength{\columnseprule}{0.5pt}
\setlength{\columnsep}{20pt}
%%%%%%%%%%%%%%%%%%%%%%%%
%%%%%%%%%%%%%%%%%%%%%%%%
%% PDF support

\usepackage[pdftex]{color,graphicx}
% %% Hyper-refs
\usepackage[pdftex]{hyperref} % for printing
% \usepackage[pdftex,bookmarks,colorlinks]{hyperref} % for screen

%% \newif\ifPDF
%% \ifx\pdfoutput\undefined\PDFfalse
%% \else\ifnum\pdfoutput > 0\PDFtrue
%%      \else\PDFfalse
%%      \fi
%% \fi

%% \ifPDF
%%   \usepackage[T1]{fontenc}
%%   \usepackage{aeguill}
%%   \usepackage[pdftex]{graphicx,color}
%%   \usepackage[pdftex]{hyperref}
%% \else
%%   \usepackage[T1]{fontenc}
%%   \usepackage[dvips]{graphicx}
%%   \usepackage[dvips]{hyperref}
%% \fi

%%%%%%%%%%%%%%%%%%%%%%%%


%%%%%%%%%%%%%%%%%%%%%%%%
%% Math
\usepackage{amsmath,amsfonts,amssymb}
% para usar R de Real do jeito que o povo gosta
\usepackage{amsfonts} % \mathbb
% para usar as letras frescas como L de Espaco das Transf Lineares
% \usepackage{mathrsfs} % \mathscr

% Oferecer seno e tangente em pt, com os comandos usuais.
\providecommand{\sin}{} \renewcommand{\sin}{\hspace{2pt}\mathrm{sen}}
\providecommand{\tan}{} \renewcommand{\tan}{\hspace{2pt}\mathrm{tg}}

% dt of integrals = \ud t
\newcommand{\ud}{\mathrm{\ d}}
%%%%%%%%%%%%%%%%%%%%%%%%



\begin{document}

%%%%%%%%%%%%%%%%%%%%%%%%
%% Título e cabeçalho
%\noindent\parbox[c]{.15\textwidth}{\includegraphics[width=.15\textwidth]{logo}}\hfill
\parbox[c]{.825\textwidth}{\raggedright%
  \sffamily {\LARGE

Lista: Produtos cartesianos, Relações e Funções

\par\bigskip}
%{Estácio -- PRONATEC\par} 
{Prof: Felipe Figueiredo\par}
{\url{http://sites.google.com/site/proffelipefigueiredo}}

\vspace{1cm}
}
%%%%%%%%%%%%%%%%%%%%%%%%


%%%%%%%%%%%%%%%%%%%%%%%%
\begin{enumerate}
\item Localize graficamente os seguintes pares ordenados no plano
  cartesiano:
  \begin{enumerate}
  \item $(1,2)$
  \item $(2,1)$
  \item $(0,-1)$
  \item $(0,0)$
  \item $(-2,-2)$
  \item $(-1,2)$
  \end{enumerate}

\item Dados os conjuntos $A=\{1,3\}$, $B=\{-1,0\}$,
  $C=\{-2,3\}$, determine o produto cartesiano pedido
  \begin{enumerate}
  \item $A \times B$
  \item $B \times A$
  \item $B \times C$
  \item $A \times C$
  \item $A^2$ (dica: $A^2 = A \times A$)
  \item $A \times B \times C$ 

    (dica: o produto de três conjuntos produz ternos ordenados
    $(x,y,z)$ com $x \in A, y \in B, z \in C$)

  \item $B^3$ (dica: veja as dicas dos itens (e) e (f))
  \end{enumerate}

\item Dados os conjuntos $A=\{1,2,3,4\}$, $B=\{-3,-1,0,2\}$, determine
  o conjunto $R$ (de pares ordenados) que representa cada uma das
  relações abaixo:

Sugestão: fazer os diagramas com setas

  \begin{enumerate}
  \item $R=\{(x,y) \in A \times B | x = 1\}$
  \item $R=\{(x,y) \in A \times B | x=2, y=0\}$
  \item $R=\{(x,y) \in B \times A | y = 1 \}$
  \item $R=\{(x,y) \in A \times B | y = 1 \}$
  \item $R=\{(x,y) \in B \times A | x < 0\}$
  \item $R=\{(x,y) \in B \times A | y \le 2x\}$
  \item $R=\{(x,y) \in A \times B | x + y = 0\}$
  \item $R=\{(x,y) \in A \times B | x + 2y > 1\}$
  \item $R=\{(x,y) \in A \times B |$ $x$ é um número primo, $y$ é par $\}$
  \end{enumerate}

\item Para cada uma das relações do exercício 3, determine o domínio
  $D(R)$ e a imagem $Im(R)$

Sugestão: fazer os diagramas com setas

\item Para cada uma das relações do exercício 3, determine o conjunto
  (de pares ordenados) da relação inversa $R^{-1}$

\item Dentre as relações do exercício 5, determine se é ou não uma
  função (Obs: Lembre-se que cada uma relação inversa $R^{-1}$ é, em
  si, uma relação).

\end{enumerate}
\end{document}
