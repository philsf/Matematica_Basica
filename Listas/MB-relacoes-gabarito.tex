%\documentclass[pdftex,a4paper]{article}
\documentclass[a4paper]{article}
%%classes: article, report, book, proc, amsproc

%%%%%%%%%%%%%%%%%%%%%%%%
%% Misc
% para acertar os acentos
\usepackage[brazilian]{babel} 
%\usepackage[portuguese]{babel} 
% \usepackage[english]{babel}
% \usepackage[T1]{fontenc}
% \usepackage[latin1]{inputenc}
\usepackage[utf8]{inputenc}
\usepackage{indentfirst}
\usepackage{fullpage}
% \usepackage{graphicx} %See PDF section
\usepackage{multicol}
\setlength{\columnseprule}{0.5pt}
\setlength{\columnsep}{20pt}
%%%%%%%%%%%%%%%%%%%%%%%%
%%%%%%%%%%%%%%%%%%%%%%%%
%% PDF support

\usepackage[pdftex]{color,graphicx}
% %% Hyper-refs
\usepackage[pdftex]{hyperref} % for printing
% \usepackage[pdftex,bookmarks,colorlinks]{hyperref} % for screen

%% \newif\ifPDF
%% \ifx\pdfoutput\undefined\PDFfalse
%% \else\ifnum\pdfoutput > 0\PDFtrue
%%      \else\PDFfalse
%%      \fi
%% \fi

%% \ifPDF
%%   \usepackage[T1]{fontenc}
%%   \usepackage{aeguill}
%%   \usepackage[pdftex]{graphicx,color}
%%   \usepackage[pdftex]{hyperref}
%% \else
%%   \usepackage[T1]{fontenc}
%%   \usepackage[dvips]{graphicx}
%%   \usepackage[dvips]{hyperref}
%% \fi

%%%%%%%%%%%%%%%%%%%%%%%%


%%%%%%%%%%%%%%%%%%%%%%%%
%% Math
\usepackage{amsmath,amsfonts,amssymb}
% para usar R de Real do jeito que o povo gosta
\usepackage{amsfonts} % \mathbb
% para usar as letras frescas como L de Espaco das Transf Lineares
% \usepackage{mathrsfs} % \mathscr

% Oferecer seno e tangente em pt, com os comandos usuais.
\providecommand{\sin}{} \renewcommand{\sin}{\hspace{2pt}\mathrm{sen}}
\providecommand{\tan}{} \renewcommand{\tan}{\hspace{2pt}\mathrm{tg}}

% dt of integrals = \ud t
\newcommand{\ud}{\mathrm{\ d}}
%%%%%%%%%%%%%%%%%%%%%%%%



\begin{document}

%%%%%%%%%%%%%%%%%%%%%%%%
%% Título e cabeçalho
%\noindent\parbox[c]{.15\textwidth}{\includegraphics[width=.15\textwidth]{logo}}\hfill
\parbox[c]{.825\textwidth}{\raggedright%
  \sffamily {\LARGE

Gabarito: Produtos cartesianos, Relações e Funções

\par\bigskip}
%{Estácio -- PRONATEC\par} 
%{Curso: Informática / Informática para Internet\par}
{Prof: Felipe Figueiredo\par}
{\url{http://sites.google.com/site/proffelipefigueiredo}}
}

Versão: \verb|20141124|

%%%%%%%%%%%%%%%%%%%%%%%%


%%%%%%%%%%%%%%%%%%%%%%%%
\begin{enumerate}

\item 

\item \ %Dados os conjuntos $A=\{1,3\}$, $B=\{-1,0\}$,
%  $C=\{-2,3\}$, determine o produto cartesiano pedido

  \begin{enumerate}
  \item $A \times B = \{(1,-1), (1,0), (3,-1), (3,0)\}$
  \item $B \times A = \{(-1,1), (-1,3), (0,1), (0,3)\}$
  \item $B \times C = \{(-1,-2), (-1,3), (0,-2), (0,3)\}$
  \item $A \times C = \{(1,-2), (1,3), (3,-2), (3,3)\}$
  \item $A^2 = \{(1,1), (1,3), (3,1), (3,3)\}$
  \item $A \times B \times C = \{(1,-1,-2), (1,-1,3), (1,0,-2),
    (1,0,3), (3,-1,-2), (3,-1,3), (3,0,-2), (3,0,3)\}$
  \item $B^3 = \{(-1,-1,-1), (-1,-1,0), (-1,0,-1), (-1,0,0),
    (0,-1,-1), (0,-1,0), (0,0,-1), (0,0,0)\}$
  \end{enumerate}

\item \ % Dados os conjuntos $A=\{1,2,3,4\}$, $B=\{-3,-1,0,2\}$, determine
  % o conjunto $R$ (de pares ordenados) que representa cada uma das
  % relações abaixo:

  \begin{enumerate}
  \item $R=\{(1,-3), (1,-1), (1,0), (1,2)\}$ %$R=\{x \in A, y \in B | x = 1\}$
  \item $R=\{(2,0)\}$ %$R=\{x \in A, y \in B | x=2, y=0\}$
  \item $R=\{(-3,1), (-1,1), (0,1), (2,1)\}$ %$R=\{x \in B, y \in A | y = 1 \}$
  \item $R=\{\ \}=\emptyset$ %$R=\{x \in A, y \in B | y = 1 \}$
  \item $R=\{(-3,1), (-3,2), (-3,3), (-3,4), (-1,1), (-1,2), (-1,3), (-1,4)\}$ %$R=\{x \in B, y \in A | x \le 0\}$
  \item $R=\{(2,1), (2,2), (2,3), (2,4)\}$ %$R=\{x \in B, y \in A | y \le 2x\}$
  \item $R=\{(1,-1), (3,-3)\}$ %$ $R=\{x \in A, y \in B | x + y = 0\}$
  \item $R=\{(1,2), (2,0), (2,2), (3,0), (3,2), (4,-1), (4,0),
    (4,2)\}$ %$R=\{x \in A, y \in B | x + 2y \ge 1\}$
  \item $R=\{(2,0), (2,2), (3,0), (3,2)\}$ %$R=\{x \in A, y \in B |$ $x$ é um número primo, $y$ é par $\}$
  \end{enumerate}

\item \ % Para cada uma das relações do exercício 3, determine o domínio
  % $D(R)$ e a imagem $Im(R)$

  \begin{enumerate}
  \item $D(R) = \{1\}$, $Im(R)=B$
  \item $D(R) = \{2\}$, $Im(R)=\{0\}$
  \item $D(R) = B$, $Im(R)=\{1\}$
  \item $D(R) = \emptyset$, $Im(R)=\emptyset$
  \item $D(R) = \{-3,-1\}$, $Im(R)=A$
  \item $D(R) = \{2\}$, $Im(R)=A$
  \item $D(R) = \{1,3\}$, $Im(R)=\{-1,-3\}$
  \item $D(R) = A$, $Im(R)=\{-1,0,2\}$
  \item $D(R) = \{2,3\}$, $Im(R)=\{0,2\}$
  \end{enumerate}

\newpage

\item \ %Para cada uma das relações do exercício 3, determine o conjunto
%  (de pares ordenados) da relação inversa $R^{-1}$
  \begin{enumerate}
  \item $R^{-1} = \{(-3,1), (-1,1), (0,1), (2,1)\}$
  \item $R^{-1} = \{(0,2)\}$
  \item $R^{-1} = \{(1,-3), (1,-1), (1,0), (1,2)\}$
  \item $R^{-1} = \{\ \} = \emptyset$
  \item $R^{-1} = \{(1,-3), (2,-3), (3,-3), (4,-3), (1,-1), (2,-1), (3,-1), (4,-1)\}$
  \item $R^{-1} = \{(1,2), (2,2), (3,2), (4,2)\}$
  \item $R^{-1} = \{(-1,1), (-3,3)\}$
  \item $R^{-1} = \{(2,1), (0,2), (2,2), (0,3), (2,3), (-1,4), (0,4), (2,4)\}$
  \item $R^{-1} = \{(0,2), (2,2), (0,3), (2,3)\}$
  \end{enumerate}

\item \ %Dentre as relações do exercício 5, determine se é ou não uma
%  função
  \begin{enumerate}
  \item é função
  \item não é função
  \item não é função
  \item é função
  \item não é função
  \item é função
  \item não é função
  \item não é função
  \item não é função
  \end{enumerate}

\end{enumerate}

\end{document}
