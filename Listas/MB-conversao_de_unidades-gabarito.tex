%\documentclass[pdftex,a4paper]{article}
\documentclass[a4paper]{article}
%%classes: article, report, book, proc, amsproc

%%%%%%%%%%%%%%%%%%%%%%%%
%% Misc
% para acertar os acentos
\usepackage[brazilian]{babel} 
%\usepackage[portuguese]{babel} 
% \usepackage[english]{babel}
% \usepackage[T1]{fontenc}
% \usepackage[latin1]{inputenc}
\usepackage[utf8]{inputenc}
\usepackage{indentfirst}
\usepackage{fullpage}
% \usepackage{graphicx} %See PDF section
\usepackage{multicol}
\setlength{\columnseprule}{0.5pt}
\setlength{\columnsep}{20pt}
%%%%%%%%%%%%%%%%%%%%%%%%
%%%%%%%%%%%%%%%%%%%%%%%%
%% PDF support

\usepackage[pdftex]{color,graphicx}
% %% Hyper-refs
\usepackage[pdftex]{hyperref} % for printing
% \usepackage[pdftex,bookmarks,colorlinks]{hyperref} % for screen

%% \newif\ifPDF
%% \ifx\pdfoutput\undefined\PDFfalse
%% \else\ifnum\pdfoutput > 0\PDFtrue
%%      \else\PDFfalse
%%      \fi
%% \fi

%% \ifPDF
%%   \usepackage[T1]{fontenc}
%%   \usepackage{aeguill}
%%   \usepackage[pdftex]{graphicx,color}
%%   \usepackage[pdftex]{hyperref}
%% \else
%%   \usepackage[T1]{fontenc}
%%   \usepackage[dvips]{graphicx}
%%   \usepackage[dvips]{hyperref}
%% \fi

%%%%%%%%%%%%%%%%%%%%%%%%


%%%%%%%%%%%%%%%%%%%%%%%%
%% Math
\usepackage{amsmath,amsfonts,amssymb}
% para usar R de Real do jeito que o povo gosta
\usepackage{amsfonts} % \mathbb
% para usar as letras frescas como L de Espaco das Transf Lineares
% \usepackage{mathrsfs} % \mathscr

% Oferecer seno e tangente em pt, com os comandos usuais.
\providecommand{\sin}{} \renewcommand{\sin}{\hspace{2pt}\mathrm{sen}}
\providecommand{\tan}{} \renewcommand{\tan}{\hspace{2pt}\mathrm{tg}}

% dt of integrals = \ud t
\newcommand{\ud}{\mathrm{\ d}}
%%%%%%%%%%%%%%%%%%%%%%%%



\begin{document}

%%%%%%%%%%%%%%%%%%%%%%%%
%% Título e cabeçalho
%\noindent\parbox[c]{.15\textwidth}{\includegraphics[width=.15\textwidth]{logo}}\hfill
\parbox[c]{.825\textwidth}{\raggedright%
  \sffamily {\LARGE

Gabarito: Conversão de Unidades

\par\bigskip}
%{Estácio -- PRONATEC\par} 
%{Curso: Informática / Informática para Internet\par}
{Prof: Felipe Figueiredo\par}
{\url{http://sites.google.com/site/proffelipefigueiredo}}
}

Versão: \verb|20141124|

%%%%%%%%%%%%%%%%%%%%%%%%


%%%%%%%%%%%%%%%%%%%%%%%%
\begin{enumerate}
\item %Faça as seguintes conversões de unidades de comprimento.
  \begin{enumerate}
  \item $10$m %1 dam para m
  \item $10^2$cm %1 m para cm
  \item $10^3\mu$m %1 mm para $\mu$m
  \item $10^4$dm %1 km para dm
  \item $2\cdot 10^6$mm %2 km para mm
  \item $15\cdot 10^{-3}$Mm %15 km para Mm
  \item $2200 \cdot 10^3 = 22 \cdot 10^5$nm %2200 $\mu$m para nm
  \end{enumerate}
\item %Faça as seguintes conversões de unidades de área.
  \begin{enumerate}
  \item $10^2$m$^2$ %1 dam$^2$ para m$^2$
  \item $10^4$cm$^2$ %1 m$^2$ para cm$^2$
  \item $10^6\mu$m$^2$ %1 mm$^2$ para $\mu$m$^2$
  \item $10^8$dm$^2$ %1 km$^2$ para dm$^2$
  \item $2\cdot 10^{12}$mm$^2$ %2 km$^2$ para mm$^2$
  \item $15\cdot 10^{-6}$Mm$^2$ %15 km$^2$ para Mm$^2$
  \item $2200 \cdot 10^6 = 22 \cdot 10^8$nm$^2$ %2200 $\mu$m$^2$ para nm$^2$
  \end{enumerate}

\item %Faça as seguintes conversões de unidades de volume.
  \begin{enumerate}
  \item $10^3$m$^3$ %1 dam$^3$ para m$^3$
  \item $10^6$cm$^3$ %1 m$^3$ para cm$^3$
  \item $10^9\mu$m$^3$ %1 mm$^3$ para $\mu$m$^3$
  \item $10^{12}$km$^3$ %1 km$^3$ para dm$^3$
  \item $2\cdot 10^{18}$mm$^3$ %2 km$^3$ para mm$^3$
  \item $15\cdot 10^{-9}$Mm$^3$ %15 km$^3$ para Mm$^3$
  \item $2200 \cdot 10^9 = 22 \cdot 10^{11}$nm$^3$ %2200 $\mu$m$^3$ para nm$^3$
  \end{enumerate}


\end{enumerate}

\end{document}
